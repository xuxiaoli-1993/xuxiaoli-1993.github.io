\documentclass[11pt, letterpaper]{article}
\usepackage{resume}  % use my resume style sheet
\usepackage[none]{hyphenat}
\usepackage{microtype}

% \fancyhead[L]{\footnotesize \bfseries Homework 3 \quad \bfseries CS6530 Machine Learning} % set footer
% \fancyfoot[R]{\footnotesize \thepage} % set footer
\fancyfoot[C]{\footnotesize \thepage \; of  \, \pageref{LastPage}}
\geometry{head=1in, bottom=1in, left=1in, right=1in} % set margin
\allowdisplaybreaks  % allow cross-page equations


%--------- document starts here -------------------------
\begin{document}

\begin{tabular}{@{} p{.5\textwidth} p{.5\textwidth} @{}}
   \multirow{4}{*}{{\textbf{\huge XUXIAO LI}}} & 
    \href{https://www.linkedin.com/in/xuxiaoli1993}{https://www.linkedin.com/in/xuxiaoli1993} \\
     & \href{https://xuxiaoli-1993.github.io}{https://xuxiaoli-1993.github.io} \\
     & xuxiao.li@utah.edu \\
     & 8012096239
\end{tabular}

\vspace{24pt}

\textbf{Ph.D. with 5 years' experience in physics-based modeling and computation}
\begin{itemize}[leftmargin=*, labelsep=5mm]
   \item Expertise in computational fluid dynamics (CFD), heat transfer, and material science.
   \item Experienced with maintaining, developing, and validating large scientific code.
   \item Collaborated with experimentalists and modelers, coauthored 6 journal articles.
\end{itemize}

\vspace{12pt}

\textbf{EDUCATION}

\fullrule

\textbf{University of Utah} \hfill Salt Lake City, Utah, USA

Ph.D., Mechanical Engineering  \hfill Jan. 2021

M.S., Mechanical Engineering  \hfill May 2019

Area of Focus: thermal-fluid and metallurgical modeling of laser welding and 3D printing.

Relevant Coursework: Thermodynamics, Finite Elements, Machine Learning.

\vskip 6pt

\textbf{Tongji University} 
\hfill
Shanghai, China

B.S., Aircraft Manufacturing Engineering \hfill June 2015

\vskip 9pt

% \textbf{Relevant Coursework}
% 
% \vspace{6pt}
% 
% \begin{tabular}{@{} p{.45\textwidth} p{.3\textwidth} p{.25\textwidth} @{}}
%    Finite Elements & Heat Transfer & Thermodynamics \\
%    Computational Fluid Dynamics & Turbulence & Kinetics \\
%    Machine Learning & Radiation & Optics
% \end{tabular}
% \vspace{3pt}

\textbf{TECHNICAL SKILLS}

\fullrule

\begin{tabular}{@{} l l l l l l l @{}}
   Fortran & Matlab & Linux & MPI & COMSOL & Git & Latex \\
   C/C++ & Python & SLURM & OpenMP & Abaqus & Tecplot & Profiling
\end{tabular}

\vspace{9pt}

\textbf{RESEARCH EXPERIENCE}

\fullrule

\textbf{CFD Software Development} \hfill Nov. 2016 -- Present
\begin{itemize}[leftmargin=*, labelsep=5mm]
   \item Maintaining a legacy code (in Fortran, over 25000 lines), General Equation Mesh Solver
      (GEMS), developed for general partial differential equations with finite volume method,
      unstructured mesh, and MPI parallelization.
   \item Developed 5 new modules (level-set and finite-difference method, over 15000 lines) to
      enable new features for multiphase flow, free-surface tracking, and fluid-structure
      interaction.
   \item Conducted systematic code benchmarking with over 15 test problems. Wrote documentation and
      trained lab members to use the code for different applications.
\end{itemize}

\vspace{3pt}

\textbf{Transient Dynamics in Laser Welding} \hfill Nov. 2018 -- Oct. 2020
\begin{itemize}[leftmargin=*, labelsep=5mm]
   \item Developed a multiphysics model that simulates the laser absorption, melting and
      solidification, molten metal flow, evaporation and condensation, thermal-capillary forces, gas
      dynamics, and interface evolution in laser welding processes.
   \item Synthesized data from X-ray imaging experiments (collaboration) to validate, calibrate, and
      improve the model. Quantified the physical forces and thermal field in the molten metal.
   \item Provided mechanism explanations on the relationship between process parameters (laser power
      and scanning speed) and pore (defect) formation.
\end{itemize}

\vspace{3pt}
\textbf{Powder-gas Interaction in Laser Powder Bed Fusion} \hfill Sep. 2019 -- Dec. 2020
\begin{itemize}[leftmargin=*, labelsep=5mm]
   \item Implemented a Lagrangian particle tracking module and integrated it into the welding model
      to simulate the powder motion in laser powder bed fusion processes.
   % \item Validated simulation results by first-principle of fluid mechanics and X-ray experiments.
   \item Identified characteristic modes of powder-gas interaction by quantifying the gas flow
      surrounding the powder particles and the forces on particle surfaces.
   \item Predicted the effects of the ambient pressure on the gas flow, powder-gas interaction, and
      powder behaviors (velocity, temperature, and ejecting angle).
\end{itemize}

\vspace{3pt}

\textbf{Machine Learning for Laser Absorption by Metal Surface} \hfill Jan. 2020 -- April 2020
\begin{itemize}[leftmargin=*, labelsep=5mm]
   \item Extracted laser absorption distribution on the surface of molten metal from the laser
      welding model, as the training and validation data sets.
   \item Applied convolutional neural network algorithms using Tensorflow to predict the laser
      absorption distribution for random molten metal surfaces in laser welding.
\end{itemize}

\vspace{3pt}

\textbf{Metallurgical Modeling for Direct Laser Deposition} \hfill Nov. 2016 -- May 2018
\begin{itemize}[leftmargin=*, labelsep=5mm]
   \item Developed a process model that simulates the thermal history and surface evolution in
      direct laser deposition processes.
   \item Implemented a cellular automata algorithm with a hybrid MPI-OpenMP parallelization (in
      C/C++) to simulate the grain melting, nucleation, and growth. 
   \item Conducted simulations to identify nucleation conditions to tailor distinct grain
      morphology, e.g., small equiaxed grains and large columnar grains.
\end{itemize}

\vspace{3pt}

\textbf{Laser Absorption by Powder Bed} \hfill Sep. 2015 -- Aug. 2016
\begin{itemize}[leftmargin=*, labelsep=5mm]
   \item Implemented a rain-dropping algorithm to generate randomly packed beds of powders as in
      typical metal 3D printing processes.
   \item Implemented the ray-tracing algorithm to simulate the multiple reflections of a laser beam
      on the surfaces of powders.
   %\item Conducted parametric studies on the effects of powder size, powder bed thickness, and powder
   %   material on the laser absorption distribution within the powder bed.
\end{itemize}

\vspace{9pt}

\textbf{SELECTED PUBLICATIONS} \hfill (full list:  
\href{https://xuxiaoli-1993.github.io/publications.html}
{https://xuxiaoli-1993.github.io/publications.html})

\fullrule

\begin{enumerate}[leftmargin=*, labelsep=4mm]
   \item \textbf{Li, X.}, Zhao, C., Sun, T., Tan, W., 2020. Revealing transient powder-gas
      interaction in laser powder bed fusion process through multi-physics modeling and high-speed
      synchrotron x-ray imaging. Additive Manufacturing, 35, p.101362. 

   \item Zhao C., Parab, N.D., \textbf{Li, X.}, Fezzaa, K., Tan, W., Rollett, A.D., Sun. T., 2020.
      Critical instability at moving keyhole tip generates porosity in laser melting. Science,
      370(6520), pp.1080-1086.

   \item Kouraytem, N., \textbf{Li, X.}, Cunningham, R., Zhao, C., Parab, N., Sun, T., Rollett,
      A.D., Spear, A.D., Tan, W., 2019. Effect of laser-matter interaction on molten pool flow and
      keyhole dynamics. Physical Review Applied, 11(6), p.064054.

   \item Zhao, C., Guo, Q., \textbf{Li, X.}, Parab, N., Fezzaa, K., Tan, W., Chen, L., Sun, T.,
      2019. Bulk-explosion-induced metal spattering during laser processing. Physical Review X,
      9(2), p.021052.

   \item \textbf{Li, X.}, Tan, W., 2018. Numerical investigation of effects of nucleation mechanisms
      on grain structure in metal additive manufacturing. Computational Material Science, 153,
      pp.159-169.

   \item Herriott, C.F., \textbf{Li, X.}, Kouraytem, N., Tari, V., Tan, W., Anglin, B.S., Rollett, A.D.,
      Spear, A.D., 2018. A multi-scale, multi-physics modeling framework to predict spatial
      variation of properties in additive-manufactured metals. Modelling and Simulation in
      Materials Science and Engineering, 27, p. 025009.
\end{enumerate}


\end{document}






