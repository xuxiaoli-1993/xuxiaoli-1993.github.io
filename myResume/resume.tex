\documentclass[11pt, letterpaper]{article}
\usepackage{resume}  % use my resume style sheet
\usepackage[none]{hyphenat}
\usepackage{microtype}

% \fancyhead[L]{\footnotesize \bfseries Homework 3 \quad \bfseries CS6530 Machine Learning} % set footer
% \fancyfoot[R]{\footnotesize \thepage} % set footer
\fancyfoot[C]{\footnotesize \thepage \; of  \, \pageref{LastPage}}
\geometry{head=1in, bottom=1in, left=1in, right=1in} % set margin
\allowdisplaybreaks  % allow cross-page equations


%--------- document starts here -------------------------
\begin{document}

\begin{tabular}{@{} p{.5\textwidth} p{.5\textwidth} @{}}
   \multirow{4}{*}{{\textbf{\huge XUXIAO LI}}} & 
    \href{https://www.linkedin.com/in/xuxiaoli1993}{https://www.linkedin.com/in/xuxiaoli1993} \\
     & \href{https://xuxiaoli-1993.github.io}{https://xuxiaoli-1993.github.io} \\
     & xuxiao.li@utah.edu \\
     & 8012096239
\end{tabular}

% \begin{center}
%    \textbf{\Large XUXIAO LI}
% 
%    801-209-6239 \textbar{} xuxiao.li@utah.edu
% 
%    LinkedIn:
%    \href{https://www.linkedin.com/in/xuxiaoli1993}{https://www.linkedin.com/in/xuxiaoli1993} \\
% 
%    Personal Website: \href{https://xuxiaoli-1993.github.io}{https://xuxiaoli-1993.github.io}
%    \paraspace
% \end{center}

\vspace{24pt}

\textbf{Ph.D. with five years' experience in physics-based modeling and computation}
\begin{itemize}[leftmargin=*, labelsep=5mm]
   \item Expertise in computational fluid dynamics (CFD), heat transfer, and material science.
   \item Experienced with maintaining, extending, and developing large scientific code.
   \item Collaborated with experimentalists and modelers, coauthored multiple journal articles.
\end{itemize}

\vspace{12pt}

\textbf{EDUCATION}

\fullrule

\textbf{University of Utah} \hfill Salt Lake City, Utah, USA

Ph.D., Mechanical Engineering  \hfill Jan. 2021

M.S., Mechanical Engineering  \hfill May 2019

Area of Focus: thermal-fluid modeling of metal 3D printing and laser welding.

Relevant Coursework: Numerical Solution of PDE, Finite Elements, Machine Learning.

\vskip 6pt

\textbf{Tongji University} 
\hfill
Shanghai, China

B.S., Aircraft Manufacturing Engineering \hfill June 2015

\vskip 9pt

% \textbf{Relevant Coursework}
% 
% \vspace{6pt}
% 
% \begin{tabular}{@{} p{.45\textwidth} p{.3\textwidth} p{.25\textwidth} @{}}
%    Finite Elements & Heat Transfer & Thermodynamics \\
%    Computational Fluid Dynamics & Turbulence & Kinetics \\
%    Machine Learning & Radiation & Optics
% \end{tabular}
% \vspace{3pt}

\textbf{TECHNICAL SKILLS}

\fullrule

\begin{tabular}{@{} l l l l l l l @{}}
   Linux & Git & Latex & Fortran & c/c++ & Python & SLURM \\
   MATLAB & Comsol & Abaqus & MPI & OpenMP & Tecplot & Profiling
\end{tabular}

\vspace{9pt}

\textbf{RESEARCH EXPERIENCE}

\fullrule

\textbf{CFD Code Development} \hfill 2016 -- Present
\begin{itemize}[leftmargin=*, labelsep=5mm]
   \item Maintaining a legacy code (in Fortran, over 25000 lines), General Equation Mesh Solver
      (GEMS), initially developed for CFD computation with finite volume method, unstructured mesh
      and MPI parallelization.
   \item Developed 5 new modules (based on the level-set method, over 15000 lines) and integrated
      them into GEMS to enable multi-phase, free-surface flow and fluid-structure interaction
      computations.
   \item Designed and conducted code benchmarking of GEMS and the new modules using over 15 fluid
      dynamic test problems. 
\end{itemize}

\vspace{3pt}

\textbf{Transient Dynamics in Laser Welding} \hfill 2018 -- Present
\begin{itemize}[leftmargin=*, labelsep=5mm]
   \item Developed a multiphysics model that simulates the laser absorption, melting and
      solidification, molten metal flow, evaporation and condensation, thermal-capillary forces, gas
      dynamics, and interface evolution in laser welding processes.
   \item Synthesized results from simulations and X-ray imaging experiments (from collaborators) to
      quantify the physical forces and thermal field in the molten metal.
   \item Provided mechanism explanations on the relationship between process parameters (laser power
      and scanning speed) and pore (defect) formation.
\end{itemize}

\vspace{3pt}
\textbf{Powder-gas Interaction in Metal 3D Printing} \hfill 2019 -- Present
\begin{itemize}[leftmargin=*, labelsep=5mm]
   \item Implemented a Lagrangian-point forcing scheme and the Discrete Element Method
      into the laser welding model to simulate the powder motion in metal 3D printing.
   \item Validated simulation results by fluid mechanics principles and X-ray experiments.
   \item Identified characteristic modes of powder-gas interaction by quantifying the gas flow
      surrounding the powder particles and the forces on particle surfaces.
   \item Predicted the effects of the ambient pressure on the gas flow (pressure and velocity
      field), powder-gas interaction, and powder behaviors (trajectory and temperature).
\end{itemize}

\vspace{3pt}

\textbf{Machine Learning for Laser Absorption by Metal Surface} \hfill Jan. -- April 2020
\begin{itemize}[leftmargin=*, labelsep=5mm]
   \item Extracted laser absorption distribution on surfaces of molten metal from the laser welding
      model, as the training and validation data sets.
   \item Applied convolutional neural network algorithms using Tensorflow to predict the laser
      absorption distribution for random molten metal surfaces in laser welding.
\end{itemize}

\vspace{3pt}

\textbf{Grain Structure Modeling for Metal 3D Printing} \hfill 2016 -- 2018
\begin{itemize}[leftmargin=*, labelsep=5mm]
   \item Developed a process model that simulates the thermal history and surface evolution in metal
      3D printing processes.
   \item Implemented a cellular automata algorithm with a hybrid MPI-OpenMP parallelization (in
      C/C++) to simulate the grain melting, nucleation, and growth. 
   \item Conducted simulations to identify nucleation conditions to tailor distinct grain
      morphology, e.g., small equiaxed grains and large columnar grains.
\end{itemize}

\vspace{3pt}

\textbf{Laser Absorption by Powder Bed} \hfill 2015 -- 2016
\begin{itemize}[leftmargin=*, labelsep=5mm]
   \item Implemented a rain-dropping algorithm to generate randomly packed beds of powders as in
      typical metal 3D printing processes.
   \item Implemented the ray-tracing algorithm to simulate the multiple reflections of a laser beam
      on the surfaces of powders.
   %\item Conducted parametric studies on the effects of powder size, powder bed thickness, and powder
   %   material on the laser absorption distribution within the powder bed.
\end{itemize}

\vspace{9pt}

\textbf{SELECTED PUBLICATIONS} \hfill (full list:  
\href{https://xuxiaoli-1993.github.io/publications.html}
{https://xuxiaoli-1993.github.io/publications.html})

\fullrule

\begin{enumerate}[leftmargin=*, labelsep=4mm]
   \item \textbf{Li, X.}, Zhao, C., Sun, T., Tan, W., 2020. Revealing transient powder-gas
      interaction in laser powder bed fusion process through multi-physics modeling and high-speed
      synchrotron x-ray imaging. Additive Manufacturing, 35, p.101362. 

   \item Zhao C., Parab, N.D., \textbf{Li, X.}, Fezzaa, K., Tan, W., Rollett, A.D., Sun. T., 2020.
      Critical instability at moving keyhole tip generates porosity in laser melting. Science,
      370(6520), pp.1080-1086.

   \item Kouraytem, N., \textbf{Li, X.}, Cunningham, R., Zhao, C., Parab, N., Sun, T., Rollett,
      A.D., Spear, A.D., Tan, W., 2019. Effect of laser-matter interaction on molten pool flow and
      keyhole dynamics. Physical Review Applied, 11(6), p.064054.

   \item Zhao, C., Guo, Q., \textbf{Li, X.}, Parab, N., Fezzaa, K., Tan, W., Chen, L., Sun, T.,
      2019. Bulk-explosion-induced metal spattering during laser processing. Physical Review X,
      9(2), p.021052.

   \item \textbf{Li, X.}, Tan, W., 2018. Numerical investigation of effects of nucleation mechanisms
      on grain structure in metal additive manufacturing. Computational Material Science, 153,
      pp.159-169.

   \item Herriott, C.F., \textbf{Li, X.}, Kouraytem, N., Tari, V., Tan, W., Anglin, B.S., Rollett, A.D.,
      Spear, A.D., 2018. A multi-scale, multi-physics modeling framework to predict spatial
      variation of properties in additive-manufactured metals. Modelling and Simulation in
      Materials Science and Engineering, 27, p. 025009.
\end{enumerate}


\end{document}






