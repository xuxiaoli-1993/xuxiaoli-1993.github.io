\documentclass[11pt, letterpaper]{article}
\usepackage{resume}  % use my resume style sheet

% \fancyhead[L]{\footnotesize \bfseries Homework 3 \quad \bfseries CS6530 Machine Learning} % set footer
% \fancyfoot[R]{\footnotesize \thepage} % set footer
\fancyfoot[C]{\footnotesize \thepage \; of  \, \pageref{LastPage}}
\geometry{head=1in, bottom=1in, left=1in, right=1in} % set margin
\allowdisplaybreaks  % allow cross-page equations


%--------- document starts here -------------------------
\begin{document}

\begin{tabular}{@{} p{.5\textwidth} p{.5\textwidth} @{}}
   \multirow{4}{*}{{\textbf{\huge XUXIAO LI}}} & 
    \href{https://www.linkedin.com/in/xuxiaoli1993}{https://www.linkedin.com/in/xuxiaoli1993} \\
     & \href{https://xuxiaoli-1993.github.io}{https://xuxiaoli-1993.github.io} \\
     & xuxiao.li@utah.edu \\
     & 8012096239
\end{tabular}

% \begin{center}
%    \textbf{\Large XUXIAO LI}
% 
%    801-209-6239 \textbar{} xuxiao.li@utah.edu
% 
%    LinkedIn:
%    \href{https://www.linkedin.com/in/xuxiaoli1993}{https://www.linkedin.com/in/xuxiaoli1993} \\
% 
%    Personal Website: \href{https://xuxiaoli-1993.github.io}{https://xuxiaoli-1993.github.io}
%    \paraspace
% \end{center}

\vspace{24pt}

\textbf{Ph.D. with five years' experience in physics-based modeling and computation}
\begin{itemize}[leftmargin=*, labelsep=5mm]
   \item Expertise in computational fluid dynamics, heat transfer, and material science.
   \item Experienced with maintaining, developing, and optimizing large scientific code.
   \item Collaborated with experimentalists and modelers, coauthored multiple journal articles.
\end{itemize}

\vspace{12pt}

\textbf{EDUCATION}

\fullrule

\textbf{University of Utah} \hfill Salt Lake City, Utah, USA

Ph.D., Mechanical Engineering  \hfill Jan. 2021

M.S., Mechanical Engineering  \hfill May 2019

Area of Focus: thermal-fluid science in metal additive manufacturing and laser keyhole welding.

Relevant Coursework: Numerical Solution of PDE, Thermodynamics, Machine Learning.

\vskip 6pt

\textbf{Tongji University} 
\hfill
Shanghai, China

B.S., Aircraft Manufacturing Engineering \hfill June 2015

\vskip 9pt

% \textbf{Relevant Coursework}
% 
% \vspace{6pt}
% 
% \begin{tabular}{@{} p{.45\textwidth} p{.3\textwidth} p{.25\textwidth} @{}}
%    Finite Elements & Heat Transfer & Thermodynamics \\
%    Computational Fluid Dynamics & Turbulence & Kinetics \\
%    Machine Learning & Radiation & Optics
% \end{tabular}
% \vspace{3pt}

\textbf{TECHNICAL SKILLS}

\fullrule

\begin{tabular}{@{} l l l l l l @{}}
   Linux & Git & Latex & Fortran & c/c++ & Python \\
   MATLAB & Comsol & Abaqus & MPI & OpenMP & Profiling
\end{tabular}

\vspace{9pt}

\textbf{RESEARCH EXPERIENCE}

\fullrule

\textbf{GEMS Maintenance} \hfill 2016 -- Present
\begin{itemize}[leftmargin=*, labelsep=5mm]
   \item Maintaining a legacy code (in Fortran, over 25000 lines), General Equation Mesh Solver
      (GEMS), initially developed for turbulent reacting flow computation with unstructured mesh and
      MPI parallelization.
   \item Developed and integrated new modules (Level-Set and Ghost Fluid Method, over 10000 lines)
      into GEMS to enable multi-phase, free-surface flow and fluid-solid interaction computations.
   \item Designed and conducted simulations of over 15 benchmark fluid dynamics
      problems for systematic verification of GEMS and the new modules.
\end{itemize}

\vspace{3pt}

\textbf{Keyhole Dynamics in Laser Welding} \hfill 2018 -- Present
\begin{itemize}[leftmargin=*, labelsep=5mm]
   \item Developed a multi-physics model (based on GEMS) that simulates the laser absorption, molten
      pool flow, evaporation/condensation kinetics, thermal-capillary forces, and keyhole evolution
      in laser welding processes.
   \item Synthesized results from simulations and X-ray imaging experiments (from collaborators) to
      make estimations on the driving forces and thermal field on the keyhole.
   \item Provided mechanism explanations on the relationship between process parameters, keyhole
      oscillation, and defect formation.
\end{itemize}

\vspace{3pt}
\textbf{Powder-gas Interaction in Laser Powder Bed Fusion} \hfill 2019 -- Present
\begin{itemize}[leftmargin=*, labelsep=5mm]
   \item Implemented a Lagrangian-point forcing scheme and the Discrete Element Method
      into the laser welding model to simulate the powder motion in laser powder bed fusion
      processes.
   \item Identified characteristic modes of powder-gas interaction based on the quantification of
      the surrounding gas flow and gas-induced forces on powders.
   \item Conducted simulations to identify the effects of ambient pressure on the gas flow and
      statistics of spattered powder, e.g., ejecting angle, temperature, and velocity.
\end{itemize}

\vspace{3pt}

\textbf{Machine Learning for Laser Absorption in Keyhole} \hfill Jan. -- April 2020
\begin{itemize}[leftmargin=*, labelsep=5mm]
   \item Extracted laser absorption distribution on keyhole's surface from the laser welding
      model as the training and validation data sets.
   \item Applied convolutional neural network algorithms using Tensorflow to predict laser
      absorption for random keyhole shapes.
\end{itemize}

\vspace{3pt}

\textbf{Cellular Automata Simulation for Grain Nucleation and Growth} \hfill 2016 -- 2018
\begin{itemize}[leftmargin=*, labelsep=5mm]
   \item Developed a thermal model (based on GEMS) that simulates the heat transfer and
      temperature field in direct energy deposition (DED) processes.
   \item Implemented the Cellular Automata (CA) algorithm to simulate the grain nucleation and
      growth given the temperature field from the thermal model. Parallelized the CA algorithm with
      hybrid OpenMP and MPI.
   \item Conducted simulations to identify nucleation conditions for tailoring distinct
      grain morphology in DED processes.
\end{itemize}

\vspace{3pt}

\textbf{Laser Absorption by Powder Bed} \hfill 2015 -- 2016
\begin{itemize}[leftmargin=*, labelsep=5mm]
   \item Implemented a rain-dropping algorithm to generate randomly packed beds of powders as in
      typical laser powder bed fusion processes.
   \item Implemented the ray-tracing algorithm to model the multiple reflections of a laser beam on
      the surfaces of powders.
   \item Conducted parametric studies on the effects of powder size, powder bed thickness, and powder
      material on the laser absorption distribution within the powder bed.
\end{itemize}

\vspace{9pt}

\textbf{SELECTED PUBLICATIONS} \hfill (full list:  
\href{https://xuxiaoli-1993.github.io/publications.html}
{https://xuxiaoli-1993.github.io/publications.html})

\fullrule

\begin{enumerate}[leftmargin=*, labelsep=4mm]
   \item Herriott, C.F., \textbf{Li, X.}, Kouraytem, N., Tari, V., Tan, W., Anglin, B.S., Rollett, A.D.,
      Spear, A.D., 2018. A multi-scale, multi-physics modeling framework to predict spatial
      variation of properties in additive-manufactured metals. Modelling and Simulation in
      Materials Science and Engineering, 27, p. 025009.

   \item \textbf{Li, X.}, Tan, W., 2018. Numerical investigation of effects of nucleation mechanisms
      on grain structure in metal additive manufacturing. Computational Material Science, 153, pp.
      159-169.

   \item Kouraytem, N., \textbf{Li, X.}, Cunningham, R., Zhao, C., Parab, N., Sun, T., Rollett,
      A.D., Spear, A.D., Tan, W., 2019. Effect of laser-matter interaction on molten pool flow and
      keyhole dynamics. Physical Review Applied, 11(6), p.064054.

   \item Zhao, C., Guo, Q., \textbf{Li, X.}, Parab, N., Fezzaa, K., Tan, W., Chen, L., Sun, T.,
      2019. Bulk-explosion-induced metal spattering during laser processing. Physical Review X,
      9(2), p.021052.

   \item \textbf{Li, X.}, Zhao, C., Sun, T., Tan, W., 2020. Revealing transient powder-gas
      interaction in laser powder bed fusion process through multi-physics modeling and high-speed
      synchrotron x-ray imaging. Additive Manufacturing, 35, p.101362. 

   \item Zhao C., Parab, N.D., \textbf{Li, X.}, Fezzaa, K., Tan, W., Rollett, A.D., Sun. T., 2020.
      Critical instability at moving keyhole tip generates porosity in laser melting. Science,
      accepted. 
\end{enumerate}


\end{document}






