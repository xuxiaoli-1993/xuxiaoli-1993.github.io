\documentclass[11pt, letterpaper]{article}
\usepackage{resume}  % use my resume style sheet
\usepackage[none]{hyphenat}
\usepackage{microtype}

% \fancyhead[L]{\footnotesize \bfseries Homework 3 \quad \bfseries CS6530 Machine Learning} % set footer
% \fancyfoot[R]{\footnotesize \thepage} % set footer
\fancyfoot[C]{\footnotesize \thepage \; of  \, \pageref{LastPage}}
\geometry{head=1in, bottom=1in, left=1in, right=1in} % set margin
\allowdisplaybreaks  % allow cross-page equations


%--------- document starts here -------------------------
\begin{document}

\begin{tabular}{@{} p{.5\textwidth} p{.5\textwidth} @{}}
   \multirow{4}{*}{{\textbf{\huge XUXIAO LI}}} & 
    \href{https://www.linkedin.com/in/xuxiaoli1993}{https://www.linkedin.com/in/xuxiaoli1993} \\
     & \href{https://xuxiaoli-1993.github.io}{https://xuxiaoli-1993.github.io} \\
     & xuxiao.li@utah.edu \\
     & 8012096239
\end{tabular}

\vspace{24pt}

\textbf{Ph.D. with 5 years' experience in physics-based modeling and computation}
\begin{itemize}[leftmargin=*, labelsep=5mm]
   \item Expertise in computational fluid dynamics (CFD), heat transfer, and material science.
   \item Experienced with maintaining, developing, and validating large scientific code.
   \item Collaborated with physicists and metrologists, coauthored 6 journal articles.
\end{itemize}

\vspace{12pt}

\textbf{EDUCATION}

\fullrule

\textbf{University of Utah} \hfill Salt Lake City, Utah, USA

Ph.D., Mechanical Engineering  \hfill May 2021

M.S., Mechanical Engineering  \hfill May 2019

Area of Focus: thermal-fluid and metallurgical modeling for laser-based manufacturing.

Coursework: Thermodynamics, Finite Elements, Machine Learning.

\vskip 6pt

\textbf{Tongji University} 
\hfill
Shanghai, China

B.S., Aircraft Manufacturing Engineering (solid and structural mechanics) \hfill June 2015

\vskip 9pt

\textbf{TECHNICAL SKILLS}

\fullrule

\vskip 3pt

\begin{tabular}{@{} l l l l l l l @{}}
   Fortran & Matlab & Linux & MPI & COMSOL & Git & SLURM \\
   C/C++ & Python & Tecplot & OpenMP & Abaqus & Latex & Gmsh
\end{tabular}

\vspace{9pt}

\textbf{RESEARCH EXPERIENCE}

\fullrule

\textbf{CFD Software Development} \hfill Nov. 2016 -- Present
\begin{itemize}[leftmargin=*, labelsep=5mm]
   \item Maintaining a proprietary CFD software (in Fortran, over 25000 lines) for turbulent
      reacting flow, featuring the finite volume method, unstructured mesh, and MPI parallelization.
   \item Developed 5 new modules (level-set and finite-difference method, over 15000 lines) to grow
      capabilities for multiphase flow, free-surface tracking, and fluid-structure interaction.
   \item Identified and solved numerical issues such as boundary condition definition, the
      discretization accuracy, and parallel communication bottlenecks.
   \item Conducted systematic benchmarking with over 15 test problems. Wrote documentation and
      trained lab members to use the code for different applications.
\end{itemize}

\vspace{3pt}

\textbf{Transient Dynamics in Laser Welding} \hfill Nov. 2018 -- Oct. 2020
\begin{itemize}[leftmargin=*, labelsep=5mm]
   \item Developed a multiphysics model that simulates the laser-metal interaction, melting and
      evaporation, molten pool flow, and gas dynamics in laser welding.
   \item Coordinated with physicists and metrologists to quantitatively compare the simulation with
      X-ray imaging data. Improved the model after analyzing the discrepancies.
   \item Quantified the thermal field and fluid dynamics to understand the pore (defect) formation.
      Provided design guidance (e.g., optimizing process parameters) to mitigate defects.
\end{itemize}

\vspace{3pt}
\textbf{Powder-gas Interaction in Laser Powder Bed Fusion} \hfill Sep. 2019 -- Dec. 2020
\begin{itemize}[leftmargin=*, labelsep=5mm]
   \item Integrated a discrete element method module into the welding model to simulate the
      interaction between micron-size  powders and high-speed gas flow in metal 3D printing.
   \item Developed scripts to automate repetitive simulations. Verified the model with lift
      coefficients and Strouhal number. Validated the model with X-ray imaging data.
   \item Developed scripts to post-process large data ($>$ 5 TB). Characterized the modes of
      powder-gas interaction to improve the understanding of powder spattering (defects).
   \item Utilized simulations to investigate the applicability of a novel process design, i.e.,
      changing the ambient pressure in the build chamber. Predicted the effects of powder-gas
      behaviors.
\end{itemize}

\vspace{3pt}

%\textbf{Finite Element Analysis for Laser Bending} \hfill Sep. 2019 -- Dec. 2019
%\begin{itemize}[leftmargin=*, labelsep=5mm]
%   \item Developed a thermo-mechanical model using Abaqus to simulate the thermal stress and strain
%      evolution in the laser bending of aluminum plates.
%   \item Conducted parametric studies for process design to achieve maximum bending angle with
%      minimum power input.
%\end{itemize}

\textbf{Deep Learning for Laser Absorption by Metal Surface} \hfill Jan. 2020 -- April 2020
\begin{itemize}[leftmargin=*, labelsep=5mm]
   \item Extracted laser absorption distribution on the surface of molten metal from the laser
      welding model, as the training and validation data sets.
   \item Applied convolutional neural network algorithms using Tensorflow to predict the laser
      absorption distribution for random molten metal surfaces in laser welding.
\end{itemize}

\vspace{3pt}

\textbf{Metallurgical Modeling for Direct Laser Deposition} \hfill Nov. 2016 -- May 2018
\begin{itemize}[leftmargin=*, labelsep=5mm]
   \item Developed a process model that simulates the thermal history and surface evolution in
      direct laser deposition processes.
   \item Implemented a cellular automata algorithm with a hybrid MPI-OpenMP parallelization (in
      C/C++) to simulate the grain melting, nucleation, and growth. 
   \item Conducted simulations to identify nucleation conditions to tailor distinct grain
      morphology, e.g., small equiaxed grains and large columnar grains.
\end{itemize}

\vspace{3pt}

\textbf{Laser Absorption by Powder Bed} \hfill Sep. 2015 -- Aug. 2016
\begin{itemize}[leftmargin=*, labelsep=5mm]
   \item Implemented a rain-dropping algorithm to generate randomly packed beds of powders as in
      typical metal 3D printing processes.
   \item Implemented the ray-tracing algorithm to simulate the multiple reflections of a laser beam
      on the surfaces of powders.
   %\item Conducted parametric studies on the effects of powder size, powder bed thickness, and powder
   %   material on the laser absorption distribution within the powder bed.
\end{itemize}

\vspace{9pt}

\textbf{SELECTED PUBLICATIONS} \hfill (full list:  
\href{https://xuxiaoli-1993.github.io/publications.html}
{https://xuxiaoli-1993.github.io/publications.html})

\fullrule

\begin{enumerate}[leftmargin=*, labelsep=4mm]
   \item \textbf{Li, X.}, Zhao, C., Sun, T., Tan, W., 2020. Revealing transient powder-gas
      interaction in laser powder bed fusion process through multi-physics modeling and high-speed
      synchrotron x-ray imaging. Additive Manufacturing, 35, p.101362. 

   \item Zhao C., Parab, N.D., \textbf{Li, X.}, Fezzaa, K., Tan, W., Rollett, A.D., Sun. T., 2020.
      Critical instability at moving keyhole tip generates porosity in laser melting. Science,
      370(6520), pp.1080-1086.

   \item \textbf{Li, X.}, Tan, W., 2020. Numerical modeling of powder-gas interaction relative to
      laser powder bed fusion process. Journal of Manufacturing Science and Engineering, 143(5), pp.
      054502.

   \item Kouraytem, N., \textbf{Li, X.}, Cunningham, R., Zhao, C., Parab, N., Sun, T., Rollett,
      A.D., Spear, A.D., Tan, W., 2019. Effect of laser-matter interaction on molten pool flow and
      keyhole dynamics. Physical Review Applied, 11(6), p.064054.

   \item Zhao, C., Guo, Q., \textbf{Li, X.}, Parab, N., Fezzaa, K., Tan, W., Chen, L., Sun, T.,
      2019. Bulk-explosion-induced metal spattering during laser processing. Physical Review X,
      9(2), p.021052.

   \item \textbf{Li, X.}, Tan, W., 2018. Numerical investigation of effects of nucleation mechanisms
      on grain structure in metal additive manufacturing. Computational Material Science, 153,
      pp.159-169.

   %\item Herriott, C.F., \textbf{Li, X.}, Kouraytem, N., Tari, V., Tan, W., Anglin, B.S., Rollett, A.D.,
   %   Spear, A.D., 2018. A multi-scale, multi-physics modeling framework to predict spatial
   %   variation of properties in additive-manufactured metals. Modelling and Simulation in
   %   Materials Science and Engineering, 27, p. 025009.
\end{enumerate}


\end{document}






