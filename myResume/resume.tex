\documentclass[11pt, letterpaper]{article}
\usepackage{resume}  % use my resume style sheet

% \fancyhead[L]{\footnotesize \bfseries Homework 3 \quad \bfseries CS6530 Machine Learning} % set footer
% \fancyfoot[R]{\footnotesize \thepage} % set footer
\fancyfoot[C]{\footnotesize \thepage \; of  \, \pageref{LastPage}}
\geometry{head=1in, bottom=1in, left=1in, right=1in} % set margin
\allowdisplaybreaks  % allow cross-page equations


%--------- document starts here -------------------------
\begin{document}

\begin{tabular}{@{} p{.4\textwidth} p{.6\textwidth} @{}}
   \multirow{4}{*}{{\textbf{\huge XUXIAO LI}}} & LinkedIn:
    \href{https://www.linkedin.com/in/xuxiaoli1993}{https://www.linkedin.com/in/xuxiaoli1993} \\
     & Personal Website: \href{https://xuxiaoli-1993.github.io}{https://xuxiaoli-1993.github.io} \\
     & xuxiao.li@utah.edu \\
     & 8012096239
\end{tabular}

% \begin{center}
%    \textbf{\Large XUXIAO LI}
% 
%    801-209-6239 \textbar{} xuxiao.li@utah.edu
% 
%    LinkedIn:
%    \href{https://www.linkedin.com/in/xuxiaoli1993}{https://www.linkedin.com/in/xuxiaoli1993} \\
% 
%    Personal Website: \href{https://xuxiaoli-1993.github.io}{https://xuxiaoli-1993.github.io}
%    \paraspace
% \end{center}

\vspace{24pt}

\textbf{PhD with 5 years' experience in physics-based modeling and computation}
\begin{itemize}[leftmargin=*, labelsep=5mm]
   \item Expertise in computational aspects of fluid dynamics, heat transfer, and material science.
   \item Experienced with the simulation of manufacturing processes, e.g., welding and 3D printing.
   \item Managed both large legacy codes and commercial software.
\end{itemize}

\vspace{12pt}

\textbf{EDUCATION}

\fullrule

\textbf{Tongji University} 
\hfill
Shanghai, China

\textit{B.S./Aircraft Manufacturing Engineering} \hfill June 2015

\vspace{6pt}
\textbf{University of Utah} \hfill Salt Lake City, Utah

\textit{M.S./Mechanical Engineering} \hfill May 2019

\textit{Ph.D./Mechanical Engineering}, Advisor: Prof. Wenda Tan \hfill Expected Dec. 2020 
\vskip 6pt

\textbf{Relevant Coursework}

\vspace{6pt}

\begin{tabular}{@{} p{.45\textwidth} p{.3\textwidth} p{.25\textwidth} @{}}
   Finite Elements & Heat Transfer & Thermodynamics \\
   Computational Fluid Dynamics & Turbulence & Kinetics \\
   Machine Learning & Radiation & Optics
\end{tabular}
\vspace{9pt}

\textbf{TECHNICAL SKILLS}

\fullrule

\begin{itemize}[leftmargin=*, labelsep=3mm, itemsep=2pt, topsep=0pt]
   \item \textit{Computer Pragmatics}: Linux, Vim, Git, Latex
   \item \textit{Programming Language}: Fortran, c/c++, Python, MATLAB
   \item \textit{Commercial Software}: Comsol, Abaqus
   \item \textit{High-Performance Computing}: MPI, OpenMP, Intel Profiling Tools and Debugger
\end{itemize}

\vspace{9pt}

% \textbf{RESEARCH HIGHLIGHT}
% 
% \fullrule
% \begin{figure}[H]
%    \centering
%    \includegraphics[height=2.75in]{GraphicalSummary.png}
% \end{figure}


\textbf{RESEARCH EXPERIENCE}

\fullrule

\textbf{Laser Absorption by Powder Bed} \hfill 2015 -- 2016
\begin{itemize}[leftmargin=*, labelsep=5mm]
   \item Implemented a rain-dropping algorithm to generate randomly packed beds of powders as in
      typical laser powder bed fusion processes.
   \item Implemented the ray-tracing algorithm to model the multiple reflections of a laser beam on
      the surfaces of powders.
   \item Conducted parametric studies on the effects of powder size, powder bed thickness, and powder
      material on the laser absorption distribution within the powder bed.
\end{itemize}

\vspace{3pt}

\textbf{GEMS Maintenance} \hfill 2016 -- Now
\begin{itemize}[leftmargin=*, labelsep=5mm]
   \item Maintaining a legacy code (in Fortran, over 25000 lines), General Equation Mesh Solver
      (GEMS), for solving general partial differential equations with unstructured mesh and MPI
      parallelization.
   \item Implemented the Level-Set and Ghost Fluid Method (over 10000 lines) into separate modules,
      and integrated the new modules into GEMS to enable multi-phase, free-surface flow and
      fluid-solid interaction computations.
   \item Designed and conducted simulations of over 15 benchmark fluid dynamics
      problems for systematic verification of GEMS and the new modules.
\end{itemize}

\vspace{3pt}

\textbf{Cellular Automata Simulation for Grain Nucleation and Growth} \hfill 2016 -- 2018
\begin{itemize}[leftmargin=*, labelsep=5mm]
   \item Developed a thermal model (based on GEMS) that simulates the heat transfer and
      temperature field in direct energy deposition (DED) processes.
   \item Implemented the Cellular Automata (CA) algorithm to simulate the grain nucleation and
      growth given the temperature field from the thermal model. Parallelized the CA algorithm with
      hybrid OpenMP and MPI.
   \item Conducted simulations to identify nucleation conditions for tailoring distinct
      grain morphology in DED processes.
\end{itemize}

\vspace{3pt}

\textbf{Keyhole Dynamics in Laser Welding} \hfill 2018 -- Now
\begin{itemize}[leftmargin=*, labelsep=5mm]
   \item Developed a multi-physics model (based on GEMS) that simulates the laser absorption, molten
      pool flow, evaporation/condensation kinetics, thermal-capillary forces, and keyhole evolution
      in laser welding processes.
   \item Synthesized results from simulations and X-ray imaging experiments (from collaborators) to
      make estimations on the magnitude of the various driving forces on the keyhole.
   \item Provided mechanism explanations on the relationship between process parameters, keyhole
      oscillation, and defect formation.
\end{itemize}

\vspace{3pt}

\textbf{Powder-gas Interaction in Laser Powder Bed Fusion} \hfill 2019 -- Now
\begin{itemize}[leftmargin=*, labelsep=5mm]
   \item Implemented a Lagrangian-point forcing scheme and the Discrete Element Method upon the
      laser welding model to simulate the powder motion in laser powder bed fusion processes.
   \item Identified characteristic modes of powder-gas interaction based on the quantification of
      the surrounding gas flow and gas-induced forces on powders.
   \item Conducted simulations to identify the effects of ambient pressure on the gas flow and
      statistics of spattered powder, e.g., ejecting angle, temperature, and velocity.
\end{itemize}

\vspace{9pt}

\textbf{PUBLICATIONS}

\fullrule

\textbf{Selected Papers} \hfill complete list at 
\href{https://xuxiaoli-1993.github.io/publications.html}
{https://xuxiaoli-1993.github.io/publications.html}

\begin{enumerate}[leftmargin=*, labelsep=4mm]
   \item \textbf{Li, X.}, Tan, W., 2016. Numerical investigation of laser absorption by metal powder
      bed in selective laser sintering processes. Solid Freeform Fabrication Symposium 2016, Austin,
      TX.

   \item \textbf{Li, X.}, Tan, W., 2018. Numerical investigation of effects of nucleation mechanisms
      on grain structure in metal additive manufacturing. Computational Material Science, 153, pp.
      159-169.

   \item Kouraytem, N., \textbf{Li, X.}, Cunningham, R., Zhao, C., Parab, N., Sun, T., Rollett,
      A.D., Spear, A.D., Tan, W., 2019. Effect of laser-matter interaction on molten pool flow and
      keyhole dynamics. Physical Review Applied, 11(6), p.064054.

   \item Zhao, C., Guo, Q., \textbf{Li, X.}, Parab, N., Fezzaa, K., Tan, W., Chen, L., Sun, T.,
      2019. Bulk-explosion-induced metal spattering during laser processing. Physical Review X,
      9(2), p.021052.

   \item \textbf{Li, X.}, Zhao, C., Sun, T., Tan, W., 2020. Revealing transient powder-gas
      interaction in laser powder bed fusion process through multi-physics modeling and high-speed
      synchrotron x-ray imaging. Additive Manufacturing, 35, p.101362. 

   \item \textbf{Li, X.}, Tan, W., 2020. Numerical modeling of powder-gas interaction Relative to
      laser powder bed fusion process. Journal of Manufacturing Science and Engineering, pp. 1-26.
\end{enumerate}






\end{document}






