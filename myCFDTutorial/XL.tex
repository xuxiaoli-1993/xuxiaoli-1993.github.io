\documentclass[11pt, letterpaper]{report}
\usepackage{XL}  % use my awesome template

% \fancyhead[L]{\footnotesize \bfseries Homework 1 \quad \bfseries CS6530 Machine Learning} % set footer
% \fancyfoot[R]{\footnotesize \thepage\ of \pageref{LastPage}} % set footer
\renewcommand{\headrulewidth}{0.0pt} % horizontal line in header
\renewcommand{\footrulewidth}{0.0pt} % horizontal line in footer
\geometry{head=1in, bottom=1in, footskip=25pt} % set margin
\allowdisplaybreaks  % allow cross-page equations


%--------- document starts here -------------------------
\begin{document}

\thispagestyle{tpstyle} % set the style of title page
{\centering
\vspace*{3.5in}  % force some space
\textbf{\Huge Computational Fluid Dynamics (CFD) with General Equation Mesh Solver (GEMS): A
Tutorial}\vspace{10pt}

\textbf{\Large by Xuxiao Li}\vspace{10pt}

\textit{\large May 3, 2020}
\vspace{8pt}
\centerline{\rule{0.5\linewidth}{0.5pt}}
\vspace{8pt}
\clearpage
\pagenumbering{arabic}  % Number page from the next page
}

\tableofcontents{}

\chapter*{Prologue}
\pagestyle{plain}
In this tutorial, I will cover multiple topics on both the theoretical and practical aspects of
computational fluid dynamics (CFD). With such a rich range of contents in CFD, I will only focus on
the certain methods that I am specialized in, while general concepts will also be discussed. The
major method discussed in this tutorial will be the Finite Volume Method (FVM). FVM is a somewhat
``old school'' method compared to the more recent and advanced variants of the Finite Element Method
(FEM). However, each method has its own advantages and disadvantages when applied to specific
problems. FVM served for a long period of time as the workhorse in solving the fluid mechanics problems, while FEM was
born to solve solid mechanics problems. This is a big gap between the ``fluid'' and the ``solid''
community. Although the trend is to merge the merits from both methods (e.g., discontinuous
Galerkin), but that is still an ongoing research topic and indeed, old habits die hard.
\paraspace

Within the ``fluid'' community, there is a division between those studying compressible flow
problems and those studying incompressible flow problems. Accordingly, the CFD methods can be
divided into the density-based methods (suitable for compressible flow) and pressure-based methods
(suitable for incompressible flow). The major difference here is that the fluid velocity in
compressible flow is typically very large (large Mach number), while fluid velocity is relative
small for incompressible flows. This gap has already been filled through years of efforts.
Density-based methods can also solve incompressible flow problems using the preconditioning
technique, which gives a unified framework for solving fluid dynamics problems. This approach will
be the focus of this tutorial. 
\paraspace

In the 80's, there were several pioneers who contributed significantly to the preconditioning
methods, e.g., Eli Turkel, Bram Van Leer, and Charles Merckle. This tutorial will be focused on
Merckle's preconditioning system, and in fact, the practical part of the tutorial is made based on
one of the in-house codes developed at Merckle's research group. The in-house code is named as the
General Equation Mesh Solver (GEMS) whose main creator is Dr. Ding Li. He worked as a research
associate with Prof. Merckle, initially at the University of Tennessee, and later at Purdue
University. Dr. Li embarked on the development of GEMS about early 1999. In 2002, the version 1.0 is
completed. In 2005, he has added the Maxwell equation into GEMS and also refined multiple features
to improve the generality of the code. In a paper Dr. Li published in 2006, he demonstrated the
capability of GEMS with impressing results.
\paraspace

I feel obliged to mention how I can have the access to the GEMS code. During the time Dr. Ding Li
was at Purdue university, there was a PhD student named Shaoyi Wen who worked with Dr. Li. Shaoyi
was then advised by Prof. Yung Shin whose research group had some collaborations with Prof.
Merckle's group. Shaoyi modified GEMS code for his needs with the help of Dr. Li and published a
paper in 2010 using GEMS to solve thermal-fluid problems in direct laser deposition processes.
Thereafter, the GEMS code seemed to be made available to Prof. Shin's group. Before Shaoyi graduated
from Shin's group, he passed the GEMS code to another PhD student at Shin's group, Wenda Tan. At
this time, Dr. Li has left Purdue (I actually don't know where he went). Wenda has never made
acquaintance with Dr. Li. However, he managed to exploit the GEMS code and have three papers
published in 2013, 2014 and 2015 with it. In these papers, Wenda simulated the laser keyhole welding
processes, and with GEMS, his model incorporated multiple physics and has high fidelity. In 2015,
Wenda ceremoniously graduated from Purdue and became an assistant professor at the University of
Utah. In the fall of 2015, I was registered as a PhD student at the University of Utah and I was
advised by Wenda (Dr. Tan) since then. Therefore, I have the privilege to study the GEMS code and
apply it for my PhD research. I have been exploring the GEMS code since 2017 and still learn new
things about it today. The practical aspects of this tutorial will be a summarize of my experience
of the GEMS code.


\chapter{Mesh}
% spacing between equations and text
\setlength{\abovedisplayskip}{3pt}
\setlength{\belowdisplayskip}{3pt}











\clearpage
\bibliographystyle{IEEEtran}
\bibliography{./ref.bib}
\end{document}





