\documentclass[12pt, letterpaper]{article}
\usepackage{resume}  % use my resume style sheet

% \fancyhead[L]{\footnotesize \bfseries Homework 3 \quad \bfseries CS6530 Machine Learning} % set footer
% \fancyfoot[R]{\footnotesize \thepage} % set footer
\fancyfoot[C]{\footnotesize \thepage \; of  \, \pageref{LastPage}}
\geometry{head=1in, bottom=1in, left=1in, right=1in} % set margin
\allowdisplaybreaks  % allow cross-page equations


%--------- document starts here -------------------------
\begin{document}
\begin{center}
   \textbf{\large XUXIAO LI}

   801-209-6239 \textbar{} xuxiao.li@utah.edu

   \href{xuxiaoli-1993.github.io}{xuxiaoli-1993.github.io}
   \paraspace
\end{center}

\textbf{EDUCATION}

\fullrule

\textbf{Tongji University} 
\hfill
Shanghai, China

\textit{B.S./Aircraft Manufacturing Engineering} \hfill Jun. 2015

\vspace{6pt}
\textbf{University of Utah} \hfill Salt Lake City, Utah

\textit{M.S./Mechanical Engineering} \hfill May 2019

\textit{Ph.D./Mechanical Engineering}, Advisor: Prof. Wenda Tan \hfill Expected Dec. 2020 
\vskip 11pt

\textbf{RELEVANT COURSEWORK}

\fullrule
\vspace{3pt}
\begin{tabular}{@{} p{.375\textwidth} p{.225\textwidth} p{.4\textwidth} @{}}
   Optics & Heat Transfer & Thermodynamics \\
   Computational Fluid Dynamics & Turbulence & Kinetics \\
   Machine Learning & Radiation & Numerical Solutions of PDEs
\end{tabular}
\vspace{9pt}

\textbf{RESEARCH EXPERIENCE}

\fullrule
\textbf{Laser Absorption by a Powderbed} \hfill 2015 -- 2016
\begin{itemize}[leftmargin=*, labelsep=5mm]
   \item Implemented the rain-dropping algorithm to generate randomly packed beds of powders as in
      typical laser powder bed fusion processes.
   \item Implemented the ray-tracing algorithm to model the multiple reflections of a laser beam on
      the surfaces of powders. Optimized the ray-tracing by a tree-search algorithm for laser
      incidence location.
   \item Analyzed the laser absorption distribution within the powderbed. Conducted parametric
      studies with respect to powder size, powderbed thickness and powder material.
\end{itemize}

\vspace{3pt}

\textbf{GEMS Maintenance} \hfill 2016 -- Now
\begin{itemize}[leftmargin=*, labelsep=5mm]
   \item Self-learned a poor-documented legacy code (in Fortran, over 25000 lines), General Equation
      Mesh Solver (GEMS), for solving general conservative PDE's with general unstructured mesh and
      MPI parallelization.
   \item Documented the methodology of GEMS. Designed and conducted multiple benchmark CFD
      simulations for the verification of GEMS.
   \item Modified subroutines for flux computation based on recent publications from
      original developers of GEMS. Achieved improved accuracy for unsteady problems.
   \item Added multiple modules (over 10000 lines) to enable multi-phase flow computations based on
      the level-set method, named as Awkward Level-Set GEMS (ALSGEMS). Designed and conducted
      multiple benchmark simulations for the verification of ALSGEMS. 
\end{itemize}

\vspace{3pt}

\textbf{Cellular Automata Simulation for Grain Nucleation and Growth} \hfill 2016 -- 2018
\begin{itemize}[leftmargin=*, labelsep=5mm]
   \item Developed a thermal model to simulate the heat conduction and temperature field in direct
      laser deposition processes based on the GEMS code.
   \item Implemented the Cellular Automata (CA) algorithm to simulate the grain nucleation and
      growth with the temperature output from the thermal model. Parallelized the CA algorithm with
      OpenMP. Implemented a dynamic scheduling scheme to alleviate computational cost.
   \item Analyzed the characteristics of the simulated grain structure with MTEX. Validated the
      simulation results with both analytical models and EBSD experiments in literature.
   \item Conducted numerical experiments on the effects of nucleation conditions on the grain size,
      shape and texture in the builds of direct laser deposition processes.
\end{itemize}

\vspace{3pt}

% \textbf{Keyhole Dynamics in Laser Welding} \hfill 2018 -- Now

\vspace{-3pt}
\paraspace

\textbf{PUBLICATION}

\fullrule

\begin{itemize}[leftmargin=*, labelsep=4mm]
   \item \textbf{Li, X.}, Tan, W., 2016. Numerical investigation of laser absorption by metal powder
      bed in selective laser sintering processes. Solid Freeform Fabrication Symposium 2016, Austin,
      TX.
   \item \textbf{Li, X.}, Tan, W., 2018. Numerical investigation of effects of nucleation mechanisms
      on grain structure in metal additive manufacturing. Computational Material Science, 153, pp.
      159-169.
   \item Kouraytem, N., \textbf{Li, X.}, Cunningham, R., Zhao, C., Parab, N., Sun, T., Rollett,
      A.D., Spear, A.D., Tan, W., 2019. Effect of laser-matter interaction on molten pool flow and
      keyhole dynamics. Physical Review Applied, 11(6), p.064054.
   \item Zhao, C., Guo, Q., \textbf{Li, X.}, Parab, N., Fezzaa, K., Tan, W., Chen, L., Sun, T.,
      2019. Bulk-explosion-induced metal spattering during laser processing. Physical Review X,
      9(2), p.021052.
   \item \textbf{Li, X.}, Zhao, C., Sun, T., Tan, W., 2020. Revealing transient powder-gas
      interaction in laser powder bed fusion process through multi-physics modeling and high-speed
      synchrotron x-ray imaging. Additive Manufacturing, 35, p.101362. 
   \item \textbf{Li, X.}, Tan, W., 2020. Numerical modeling of powder-gas interaction in laser
      powder bed fusion process. Journal of Manufacturing Science and Engineering, accepted.
\end{itemize}
\vspace{-3pt}
\paraspace

\textbf{TECHNICAL SKILLS}

\fullrule

\begin{itemize}[leftmargin=*, labelsep=3mm, itemsep=2pt, topsep=0pt]
   \item \textit{Computer Pragmatics}: Linux, Vim, Git, Latex
   \item \textit{Programming Language}: Fortran, c/c++, Python, MATLAB
   \item \textit{Commercial Software}: Comsol, Abaqus
   \item \textit{High Performance Computing}: MPI, OpenMP, Intel Profiling Tools and Debugger
\end{itemize}

\paraspace
\textbf{TEACHING ASISTANTSHIPS}

\fullrule
Manufacturing for Engineering Systems \hfill Fall 2016, Spring 2017, Fall 2017







\end{document}






